\documentclass[12pt]{article}
\pdfoutput=1

\usepackage{graphics,amssymb,float}
%\usepackage[usenames,dvips]{color}
\usepackage{graphicx}% Include figure files
\usepackage{subfigure}
\usepackage{color}
\usepackage{xspace}
\usepackage{placeins}
%\usepackage[color]{showkeys}
%\usepackage{epsfig}% Include figure files
\usepackage{rotating}% Include figure files
\usepackage{dcolumn}% Align table columns on decimal point
\usepackage{bm}% bold math
\usepackage{cite}
\usepackage{amsmath}
\usepackage{indentfirst} % activate indent in the first paragraph
\usepackage{epstopdf}
\usepackage{tikz}
\usetikzlibrary{shapes,arrows}

\bibliographystyle{utphys}


\textheight=22.8 truecm
\textwidth=17 truecm
\topmargin=-3mm
\voffset=-1 truecm
\hoffset=-2 truecm

\def\eg{{\it e.g.}}
\def\ie{{\it i.e.}}
\def\mgut{M_{\rm GUT}}

\def\Eq#1{Eq.~(\ref{#1})}
%\def\Eqs#2{Eqs.~(\ref{#1})--\ref{#2})}

\def\lsim{\mathrel{\raise.3ex\hbox{$<$\kern-.75em\lower1ex\hbox{$\sim$}}}}
\def\gsim{\mathrel{\raise.3ex\hbox{$>$\kern-.75em\lower1ex\hbox{$\sim$}}}}
\def\ifmath#1{\relax\ifmmode #1\else $#1$\fi}

\def\change{\marginpar{\bf text changed}}
\def\new{\marginpar{\bf new addition}}
\def\check{\marginpar{\bf check!}}

\begin{document}
%\begin{titlepage}
\begin{center}


%\vspace*{-1cm}
\begin{flushright}
preprint numbers
%LAPTH-061/12\\
%LPSC12350\\
%LPT Orsay 12-119\\
%NSF-KITP-12-236
\end{flushright}

\vspace*{1.6cm}
{\Large\bf A survey of supersymmetry\\[3mm] based on simplified-model results from the LHC} 

\vspace*{1cm}\renewcommand{\thefootnote}{\fnsymbol{footnote}}

{\large 
%B.~Dumont$^{1}$\footnote[2]{Email: dumont@lpsc.in2p3.fr},
Sabine~Kraml$^{1}$\footnote[1]{Email: sabine.kraml@lpsc.in2p3.fr},
Suchita~Kulkarni$^{1}$\footnote[2]{Email: suchita.kulkarni@lpsc.in2p3.fr},
Wolfgang Waltenberger$^{2}$\footnote[3]{Email: wolfgang.waltenberger@oeaw.ac.at}
} 

\renewcommand{\thefootnote}{\arabic{footnote}}

\vspace*{1cm} 
{\normalsize \it 
$^1\,$Laboratoire de Physique Subatomique et de Cosmologie, UJF Grenoble 1,
CNRS/IN2P3, INPG, 53 Avenue des Martyrs, F-38026 Grenoble, France\\[2mm]
$^2\,$Institut f\"ur Hochenergiephysik,  \"Osterreichische Akademie der Wissenschaften,\\ Nikolsdorfer Gasse 18, 1050 Wien, Austria\\[1mm]
}

\vspace{1cm}

\begin{abstract}
  Abstract
\end{abstract}

\end{center}

%\end{titlepage}


%%%%%%%%%%%%%%%%%%%%%%%%%%%%%%%%%%%%%%%%
\section{Introduction}
%%%%%%%%%%%%%%%%%%%%%%%%%%%%%%%%%%%%%%%%

Searches at the ATLAS and CMS experiments at the LHC show no signs of new physics whatsoever. 
After the first phase of LHC operation at centre-of-mass energies of 7--8~TeV in 2010--2012, 
the limits for the masses of supersymmetric particles, in particular of  squarks and gluinos, 
have been pushed well into the TeV range~\cite{atlas:susy:twiki,cms:susy:twiki}. 
%See \cite{atlas:susy:twiki,cms:susy:twiki} for the latest results.
Likewise, precision measurements in the flavor sector, in particular in $B$-physics, 
are well consistent with Standard Model (SM) expectations~\cite{Amhis:2012bh,lhcb:2012ct} 
and show no sign, or need, of new physics. 
At the same time the recent discovery~\cite{atlas:2012gk,cms:2012gu} of a Higgs-like particle 
with mass around 125~GeV makes the question of stability of the electroweak scale---the 
infamous gauge hierarchy problem---even  more imminent. 
Indeed, supersymmetry (SUSY) is arguably the best-motivated theory to solve the gauge 
hierarchy problem and to explain a light SM-like Higgs boson. 
So, the Higgs has very likely been discovered---but where is supersymmetry? 

Looking closely \cite{Sekmen:2011cz,Arbey:2011un,Papucci:2011wy,CahillRowley:2012kx,Dreiner:2012gx,Mahbubani:2012qq} 
one soon realizes that many of the current limits on SUSY particles are based 
on severe model assumptions, which impose particular relations between particle masses, decay 
branching rations, {\it etc}. The prime example is the interpretation of the search results within the 
Constrained Minimal Supersymmetric Standard Model (CMSSM). Another approach, which   
has become quite standard for the experiments, is to interpret the results 
within so-called Simplified Model Spectra \cite{Okawa:2011xg,cms:2013wc}.  
Simplified Model Spectra, or SMS for short, are effective-Lagrangian descriptions involving %the interactions of 
just a small number of new particles. They were designed as a useful tool for the characterization 
of new physics, see \eg\ \cite{Alwall:2008ag,Alves:2011wf}. 
On the whole, a large variety of results on searches in many different channels\footnote{Unfortunately with large variations in the level of documentation and details given on the analyses.} are available from both ATLAS and CMS, but it is quite difficult to determine whether a particular parameter point or scenario is allowed or excluded. 

In this paper, we present a method to decompose the signal of an arbitrary SUSY spectrum into simplified model  
topologies and thus test it against all the existing bounds derived in the SMS context. 
As we will show, this allows a vast survey of general and less general SUSY models, and enormously simplifies the task of designing the regions of parameter space which are still allowed, and/or have not yet been explored, by the current searches.
A scan over a 7-parameter realization of the MSSM is used as a show-case example to demonstrate the power of our method. 
Even in this example of limited complexity (as compared to, \eg, the 19-parameter phenomenological MSSM or the completely general MSSM with more than 100 free parameters) it turns out that large regions of interesting parameter space with SUSY particles below 1~TeV %survive the current bounds. 
remain unchallenged by the current bounds.

%The question what we really know about SUSY in general has been adressed in \cite{Sekmen:2011cz,Arbey:2011un,Papucci:2011wy,CahillRowley:2012kx} and some loopholes for evading the mainstream limits have been pointed out in, \eg, \cite{Dreiner:2012gx,Mahbubani:2012qq}. A recast of SMS limits onto minimal supergravity was attempted in \cite{Gutschow:2012pw}.

\subsection{Simplified models spectra philosophy and naming convention}
\def\chitz{\ensuremath{\widetilde{\chi}^0_2}}
\def\chiz{\ensuremath{\widetilde{\chi}^0_1}}
\def\chipm{\ensuremath{\widetilde{\chi}^\pm_1}}
\def\chimp{\ensuremath{\widetilde{\chi}^\mp_1}}
\def\tchi{\ensuremath{\widetilde{\chi}}}
\def\gl{\tilde{g}}
\def\sTop{\ensuremath{\tilde{t}}\xspace}
\def\st{\ensuremath{\tilde{t}}\xspace}
\def\slep{\ensuremath{\tilde{l}}\xspace}
\def\snu{\ensuremath{\tilde{\nu}}\xspace}
\def\sBottom{\tilde{b}}
\def\sb{\tilde{b}}
\def\sq{\tilde{q}}
\def\sb{\tilde{b}}
\def\st{\tilde{t}}
\def\fb{\mathrm{fb}}
\def\to{\rightarrow}
\def\first{1$^\mathrm{st}$}
\def\second{2$^\mathrm{nd}$\xspace}
\def\third{3$^\mathrm{rd}$\xspace}
\def\spacer{\hspace*{10mm} }
\definecolor{red}{rgb}{.6,.02,.02} 
%\newcommand\fixme[1]{{\begin{color}{red}FIXME #1\end{color}}}
\newcommand\fixme[1]{{\color{red}FIXME #1}}
\newcommand\model[1]{{\tt #1}}
\newcommand\url[1]{{\nopagebreak{\tt #1}}}
\newcommand\smallurl[1]{{\small \tt #1}}
\newcommand{\AlphaT}{\ensuremath{\alpha_{\mathrm{T}}}\xspace} 
\newcommand{\ATLLepStop}{ATL lep-$\tilde{t}$}
\newcommand{\ATLHadStop}{ATL had-$\tilde{t}$}
\newcommand{\SSnoHT}{SSnoH$_{\mathrm{T}}$}
\newcommand{\HsTjets}{\ensuremath{\not\!\!\mathrm{H}_{\mathrm{T}}}+jets\xspace}
\newcommand{\MTtwo}{\ensuremath{{M}_{\mathrm{T2}}}\xspace}  
\newcommand{\emu}{\ensuremath{e/\mu}\xspace}
\newcommand{\ETslash}{\ensuremath{\not\!\!\rm{E}_{\mathrm{T}}}\xspace} 
\newcommand{\ZMET}{Z+\ETslash}
\newcommand{\sigmaXBF}{\mbox{\ensuremath{\sigma\times\mathcal{B}}}\xspace}

\label{ssec:names}

The guiding principle behind the SMS decomposition is the assumption that most experimental searches
for new physics are insensitive to various specific details of the BSM model, such as the spin, charge and interactions
of the new physical states, {\it as long as they give rise to the same signal topology}. 
This assumption is clearly violated by searches which strongly rely on specific details of the
signal, such as spin correlations and properties of off-shell states. Nonetheless, 
most current experimental analyses aim for model independent constraints and can 
be interpreted within the SMS framework.
In these cases, to a first approximation, it is possible to reduce all the properties of a BSM model to its mass
spectrum, branching ratios (BRs) and production cross-sections\footnote{Clearly the calculation of branching ratios and production cross-sections depend on the specific properties of the new physical states and their interactions.
 Nonetheless, in the SMS approach all the complexity of the BSM model can be replaced
by the knowledge of its BRs, cross-sections and mass spectrum.}.
With this knowledge we can decompose the BSM signal in a series of independent signal topologies with their specific weights given
by the corresponding production cross-section times branching ratio (\sigmaXBF). Such decomposition is extremely helpful to cast the theoretical
predictions of a specific BSM model in a model-independent framework, which can then be compared against model-independent experimental searches.
The specific details of the decomposition procedure will be discussed in Sec.\ref{ssec:decomp}.


In order to clarify the above statements and set some of the notation used here, we show in Fig.\ref{fig:SMSexample} two common topologies which appear
in R-parity conserving supersymmetric models. Although the diagrams in Fig.\ref{fig:C1N2} and \ref{fig:SLSN} have a very distinct particle (and spin) content, 
most experimental searches sensitive to the "naked" diagram (SMS topology) of Fig.\ref{fig:C1N2naked} can be used to constrain the other two.

\begin{figure}[h!t]
\begin{center}
\begin{tabular}{lcr}
\subfigure[\label{fig:C1N2}$\tchi_1^+$,$\chitz$ topology]{\includegraphics[width=0.25\linewidth]{figures/C1N2.pdf}}\spacer &
\subfigure[\label{fig:SLSN}$\slep_1^+$,$\snu$ topology]{\includegraphics[width=0.25\linewidth]{figures/SLSN.pdf}}\spacer &
\subfigure[\label{fig:C1N2naked}SMS equivalent topology]{\includegraphics[width=0.27\linewidth]{figures/C1N2naked.pdf}}\spacer \\
\end{tabular}
\caption{Some examples of model-dependent signal topologies and their SMS equivalent.}
\label{fig:SMSexample}
\end{center}
\end{figure}

Since here we only consider models with a $\mathbb{Z}_2$ symmetry (such as R-Parity, T-Parity or KK-Parity),
the possible signal topologies will always consist of 
pair production of new ($\mathbb{Z}_2$-odd) particle states\footnote{Throughout this work we 
ignore BSM particles which are $\mathbb{Z}_2$-even, such as heavy higgses in the MSSM. These
cases will be discussed in future work.},
which decay as $P \to P' + $SM particles,
where $P$ and $P'$ are the parent and daughter BSM particles, respectively.
Hence all topologies will be of the form shown in Fig.\ref{fig:GenSMSTop}, consisting of two branches, each one characterized by its number of vertices and the SM particles appearing in each vertex. 
In our notation all particles appearing in the SMS topology
 (both $\mathbb{Z}_2$-even and $\mathbb{Z}_2$-odd) are on-shell. 
The case of off-shell decays are always included as 3-body decays, with no mention to off-shell states.
Therefore all the relevant information (in the SMS framework) of such a diagram can be 
reduced to three main objects:
\begin{itemize}
\item The diagram topology (number of vertices and SM final state particles in each vertex)
\item The masses of the BSM particles ($\mathbb{Z}$-odd) appearing in the diagram
\item The diagram weight (\sigmaXBF).
\end{itemize}

In order to classify the possible signal topologies we choose to label them according to the
SM states appearing in each vertex, as well as the number of vertices in each branch\footnote{Although the experimental collaborations have already introduced their
own naming convention, we choose not to use them, since they are more strongly biased by specific
BSM models.}.
To illustrate our notation, we show in Fig.\ref{fig:SMSlabel} the labeling scheme used to describe the 
diagram in Fig.\ref{fig:C1N2naked}.
Each outer pair of brackets refer to a branch ($[B_1,B_2]$). 
Inside each branch bracket, the inner brackets
correspond to the lists of SM particles coming out of each vertex.
Note that there is no mention to the intermediate BSM particles ($\tchi_1$, $\slep$,...),
what makes our method explicitly model independent. The only information kept from the BSM states
are their masses. For the SMS topology of Fig.\ref{fig:C1N2naked}, we have $[B_1,B_2]$, with $B_1 = [[l^+],[\nu]]$ and $B_2 = [[l^+,l^-]]$, 
as illustrated in Fig.\ref{fig:SMSlabel}. With the labeling scheme just introduced it becomes obvious
that the two (model dependent) diagrams in Figs.\ref{fig:C1N2} and \ref{fig:SLSN} have identical SMS topologies.

\begin{figure}[h!t]
\begin{center}
\includegraphics[width=0.35\linewidth]{figures/GeneralTop.pdf}
\caption{The general type of SMS topology considered in this document. $P_i$ label the SM final state particles.}
\label{fig:GenSMSTop}
\end{center}
\end{figure}


\begin{figure}[h!t]
\begin{center}
\includegraphics[width=0.7\linewidth]{figures/C1N2labels.pdf}
\caption{The labeling scheme adopted here applied to the diagram in Fig.\ref{fig:C1N2naked}.}
\label{fig:SMSlabel}
\end{center}
\end{figure}



\subsection{Anatomy of an SMS result}
\label{ssec:smsstatistics}


When interpreting the experimental results in the SMS framework, the
BSM model is completely fixed by its SMS topologies 
(defined by the number of vertices and the SM particles appearing in each vertex), 
the masses appearing in the topologies and the topology cross-sections.
In most cases the experimental collaborations choose a single SMS topology
(some exceptions are discussed below) and scan over the unknown masses or a subset of the masses
(with simplifying assumptions for the other masses).
Then the only unkown is the topology cross-section, which can be constrained using data.
The CMS and ATLAS collaborations typically use such scans to produce two types
of resulting plots: given a well-defined signal region, the collaborations
can report the analysis' acceptance times efficiency values ($A \times
\epsilon$), as a function of the mass parameters for an assumed SMS topology. 
Such a result is shown in Fig.~\ref{fig:smsexampleeff}. 
In the second case, a 95\% confidence level upper limit (UL) on
the SMS topology's cross section ($\sigma$) is computed, as a function
of the topology's mass parameters. A typical result can be seen in
Fig.~\ref{fig:smsexampleul}: the colors show the binned values for the upper limit on $\sigma$.
Finally, assuming theoretical ``reference'' cross sections for each mass value,
an exclusion curve is produced. Therefore, the exclusion curves can only be
applied to models with the same reference cross-section assumed by the collaboration.
Also, as the experimental searches become more complex and deviate from the simple cut and count analyses,
efficiency plots can no longer be produced \fixme{true?}. Hence this document builds upon the 
95\% upper limits on $\sigma$ -- neither the
efficiency plots nor the exclusion lines are of any relevance for this work.

\begin{figure}[ht!]
\begin{center}
\begin{tabular}{lr}
\subfigure[\label{fig:smsexampleeff}efficiency plot]{\includegraphics[width=0.5\linewidth]{figures/h_eff_T3lh_OSshape.pdf}} &
\subfigure[\label{fig:smsexampleul}upper limits on production cross
section]
{\includegraphics[width=0.5\linewidth]{figures/h_limit_T3lh_OSshape.pdf}}
\\
\end{tabular}
\caption{An example of an SMS result of the CMS collaboration, taken
from~\cite{cms:2013wc}.}
\label{fig:smsexample}
\end{center}
\end{figure}

In some particular cases, motivated by specific BSM models, the experimental collaborations
choose to apply the experimental constraints to a certain combination of SMS topologies.
The simplest example is the CMS dilepton search\cite{SUS-12-022} 
for slepton pair production and decay: $\slep^+ + \slep^- \to (l^+ \chiz) + (l^- \chiz)$,
where $\slep = \tilde{e}, \tilde{\mu}$.
Under the assumption that the selectron and smuon are degenerate, the experimental collaboration
do not constraint each flavor individually, but choose to constraint the sum of the two SMS topology
 cross-sections instead: $\sigma([[[e^+]],[[e^-]]]) + \sigma([[[\mu^+]],[[\mu^-]]])$,
where we are using the notation introduced in Sec.\ref{ssec:names}. In such cases, the experimental
constraints can only be applied under some assumptions. For the slepton case just
mentioned the assumption is that each single topology ($[[[e^+]],[[e^-]]]$ or $[[[\mu^+]],[[\mu^-]]]$)
contributes equally to the sum ($\sigma([[[e^+]],[[e^-]]]) = \sigma([[[\mu^+]],[[\mu^-]]])$). 
This condition can be relaxed if we assume that the signal efficiency for the SMS topology
with muons in the final state is higher than for electrons, so it is enough to require 
$\sigma([[[e^+]],[[e^-]]]) \leq \sigma([[[\mu^+]],[[\mu^-]]])$. 
Other examples of experimental constraints on sums of topologies can be seen in Table~\ref{tab:LHCresults}
and envolve the SMS topologies appearing in the production of electroweak inos 
decaying through sleptons and/or sneutrinos\cite{SUS-12-022,ATLAS-CONF-2013-035,ATLAS-CONF-2013-028,ATLAS-CONF-2013-036}.



Finally, we point out that it is a long-standing wish of the authors that the experimental
collaborations publish not only the 95\% upper limits, but rather the entire
likelihoods.  This would e.g. allow combinations of LHC results with
uncorrelated non-LHC measurements.  
A combination of several LHC results would still not be feasible because 
the correlations between the SMS results are unknown.
It is an even longer term vision that all ingredients that enter
into the statistical procedure are published also;
in an ideal world a user of the LHC results would be able to produce 
a likelihood for any combination of SMS results published, be they from CMS or
ATLAS.

\subsection{LHC results used}
\label{ssec:lhc}

In our subsequent results we consider the latest ATLAS and CMS analyses (with results
interpreted in terms of simplified models) for R-Parity conserving supersymmetric models. The complete list of
results used are shown in Table \ref{tab:LHCresults} in the Appendix, where we also show
the SMS topologies constrained by each analysis.
As discussed in Sec.\ref{ssec:smsstatistics}, some of the results constrain
sums of single SMS topologies and include additional assumptions about
the contribution of each topology. Whenever such results are applied to exclude a model 
we verify that the respective assumptions are satisfied.




%%%%%%%%%%%%%%%%%%%%%%%%%%%%%%%%%%%%%%%%
\section{Decomposition into simplified model topologies}
%%%%%%%%%%%%%%%%%%%%%%%%%%%%%%%%%%%%%%%%
\subsection{SMS decomposition: The procedure}
\label{ssec:decomp}

Starting point of the SMS decomposition is an SLHA file which contains the
mass spectrum and decay table of a specific MSSM parameter point.
Next, the SLHA file is passed to pythia to generate 10000 events. 
As an output format we chose the LHE file format.
Per event, we then count the SUSY particles, with the SUSY ``mothers'' (i.e.
the first SUSY particles in the decay chains) playing a special role.
Assuming $\chiz$ to always be the LSP  and taking into account conservation of
R-parity, we found the categorization scheme given in
Tab.~\ref{tab:classmothers} to be sufficient. 
Within these categories the events are further categorized based on the 
remaining SUSY particle content. Tabs.~\ref{tab:classgluinos},\ref{tab:classsquarks},\ref{tab:classstops},\ref{tab:classsbottoms},\ref{tab:classweakinos} 
discuss these subcategories. The events' requirements on the particle content is listed, in order for the event to be classified as a specific simplified model.
The requirement on the SUSY mothers and the requirement on the LSPs are not listed.
Fig.~\ref{fig:smsdecomposition} shows an overview of the whole procedure of SMS decomposition.

\begin{table}[h!t]\centering
\begin{tabular}{|c|c|c|}
\hline
Requirement on  & SMS & discussed \\
SUSY mothers  & case  & in Sec. \\
\hline
two gluinos & ``gluinos'' & Sec.~\ref{ssec:gluinocase} \\
two squarks (\first, \second gen) & ``squarks'' & Sec.~\ref{ssec:squarkcase} \\
one gluino, one squark (any generation) & ``associate'' & not discussed \\
two stops & ``stops'' & Sec.~\ref{ssec:thirdcase} \\
two sbottoms & ``sbottoms'' & Sec.~\ref{ssec:thirdcase} \\
two weakinos & ``weakinos'' & Sec.~\ref{ssec:weakinocase} \\
two sleptons & ``sleptons'' & not discussed \\
\hline
\end{tabular}
\caption{Classification of events according to SUSY mothers}
\label{tab:classmothers}
\end{table}

\begin{figure}[h!t]\centering
% \begin{tikzpicture}[anchor=mid,>=latex',line join=bevel,]
\begin{tikzpicture}[>=latex',line join=bevel,]
  \pgfsetlinewidth{1bp}
\begin{small}%
\pgfsetcolor{black}
  % Edge: gg -> T5
  \draw [->] (126.57bp,397.05bp) .. controls (135.39bp,391.92bp) and (149.05bp,384.92bp)  .. (162bp,382bp) .. controls (196.2bp,374.3bp) and (235.83bp,374.32bp)  .. (276.33bp,376.82bp);
  \definecolor{strokecol}{rgb}{0.0,0.0,0.0};
  \pgfsetstrokecolor{strokecol}
  \draw (195bp,393bp) node {qqqqV$^{(*)}$V$^{(*)}$};
  % Edge: tt -> T2tt
  \draw [->] (128.16bp,199.13bp) .. controls (154.04bp,201.76bp) and (219.44bp,208.4bp)  .. (275.35bp,214.08bp);
  \draw (195bp,218bp) node {tt};
  % Edge: ss -> T2
  \draw [->] (128.16bp,320.19bp) .. controls (154.72bp,323.02bp) and (222.91bp,330.29bp)  .. (278.9bp,336.26bp);
  \draw (195bp,339bp) node {qq};
  % Edge: lsp -> nn
  \draw [->] (36.796bp,245.22bp) .. controls (54.879bp,207.07bp) and (90.966bp,130.93bp)  .. (112.28bp,85.962bp);
  % Edge: bb -> T6ttWW
  \draw [->] (128.38bp,265.1bp) .. controls (137.42bp,262.95bp) and (150.43bp,260.18bp)  .. (162bp,259bp) .. controls (192.86bp,255.84bp) and (227.43bp,255.37bp)  .. (265.95bp,255.98bp);
  \draw (195bp,268bp) node {ttWW};
  % Edge: lsp -> gg
  \draw [->] (40.219bp,289.09bp) .. controls (58.037bp,315.53bp) and (88.262bp,360.37bp)  .. (110.67bp,393.61bp);
  % Edge: bb -> T2bb
  \draw [->] (128.44bp,270.46bp) .. controls (137.51bp,272.38bp) and (150.54bp,275.03bp)  .. (162bp,277bp) .. controls (195.61bp,282.79bp) and (233.59bp,288.29bp)  .. (273.26bp,293.71bp);
  \draw (195bp,295bp) node {bb};
  % Edge: nn -> TChiwz
  \draw [->] (143.16bp,78.656bp) .. controls (172.34bp,81.618bp) and (220.39bp,86.497bp)  .. (268.46bp,91.377bp);
  \draw (195bp,96bp) node {WZ};
  % Edge: tt -> T6bbWW
  \draw [->] (123.36bp,188.54bp) .. controls (131.1bp,177.87bp) and (145.25bp,160.99bp)  .. (162bp,153bp) .. controls (190.39bp,139.46bp) and (225.07bp,135.03bp)  .. (264.16bp,134.01bp);
  \draw (195bp,162bp) node {bbWW};
  % Edge: lsp -> ss
  \draw [->] (48.494bp,280.61bp) .. controls (63.718bp,289.14bp) and (83.605bp,300.28bp)  .. (107.01bp,313.4bp);
  % Edge: gg -> T1
  \draw [->] (124.4bp,411.38bp) .. controls (132.67bp,420.18bp) and (146.89bp,433.74bp)  .. (162bp,441bp) .. controls (195.65bp,457.17bp) and (237.5bp,464.63bp)  .. (278.93bp,469.08bp);
  \draw (195bp,470bp) node {qqqq};
  % Edge: lsp -> bb
  \draw [->] (51.834bp,268bp) .. controls (65.489bp,268bp) and (82.071bp,268bp)  .. (105.45bp,268bp);
  % Edge: nn -> TChiN2C1
  \draw [->] (133.23bp,67.857bp) .. controls (141.6bp,64.088bp) and (152.1bp,60.007bp)  .. (162bp,58bp) .. controls (191.37bp,52.042bp) and (224.6bp,50.779bp)  .. (262.48bp,51.452bp);
  \draw (195bp,67bp) node {lll};
  % Edge: lsp -> tt
  \draw [->] (46.196bp,252.46bp) .. controls (62.304bp,240.07bp) and (84.615bp,222.91bp)  .. (107.92bp,204.98bp);
  % Edge: gg -> T3
  \draw [->] (128.16bp,404.3bp) .. controls (151.79bp,407.06bp) and (208.4bp,413.67bp)  .. (263.15bp,420.06bp);
  \draw (195bp,426bp) node {qqqqV$^{(*)}$};
  % Edge: nn -> TChiSlep
  \draw [->] (122.82bp,65.896bp) .. controls (130.25bp,54.068bp) and (144.28bp,34.959bp)  .. (162bp,26bp) .. controls (184.46bp,14.642bp) and (211.16bp,9.6441bp)  .. (245.96bp,7.2855bp);
  \draw (195bp,36.5bp) node {\slep$|$\snu lll};
  % Edge: tt -> T2FVttcc
  \draw [->] (127.11bp,193.37bp) .. controls (136.07bp,189.51bp) and (149.61bp,184.31bp)  .. (162bp,182bp) .. controls (192.05bp,176.41bp) and (226.01bp,175.06bp)  .. (264.3bp,175.39bp);
  \draw (195bp,190bp) node {cc};
  % Node: TChiSlep
\begin{scope}
  \definecolor{strokecol}{rgb}{0.0,0.0,0.0};
  \pgfsetstrokecolor{strokecol}
  \draw (382bp,22bp) -- (246bp,23bp) -- (246bp,0bp) -- (382bp,0bp) -- cycle;
  \draw (314bp,11bp) node {\;\;\;\;\model{TChiSlep*} [Figs.~\ref{fig:TChiChipmSlepSlep},\ref{fig:TChiChipmSnuSlep}] \;\;};
\end{scope}
  % Node: T6ttWW
\begin{scope}
  \definecolor{strokecol}{rgb}{0.0,0.0,0.0};
  \pgfsetstrokecolor{strokecol}
  \draw (362bp,270bp) -- (266bp,270bp) -- (266bp,247bp) -- (362bp,247bp) -- cycle;
  \draw (314bp,258bp) node {\model{T6ttWW} [Fig.~\ref{fig:T6ttWW}]};
\end{scope}
  % Node: TChiwz
\begin{scope}
  \definecolor{strokecol}{rgb}{0.0,0.0,0.0};
  \pgfsetstrokecolor{strokecol}
  \draw (359bp,108bp) -- (269bp,108bp) -- (269bp,85bp) -- (359bp,85bp) -- cycle;
  \draw (314bp,96bp) node {\model{TChiwz} [Fig.~\ref{fig:TChiwz}]};
\end{scope}
  % Node: nn
\begin{scope}
  \definecolor{strokecol}{rgb}{0.0,0.0,0.0};
  \pgfsetstrokecolor{strokecol}
  \definecolor{fillcol}{rgb}{0.9,0.9,0.9};
  \pgfsetfillcolor{fillcol}
  \filldraw [opacity=1.0] (117bp,76bp) ellipse (27bp and 10bp);
  \draw (117bp,76bp) node {weakino};
\end{scope}
  % Node: bb
\begin{scope}
  \definecolor{strokecol}{rgb}{0.0,0.0,0.0};
  \pgfsetstrokecolor{strokecol}
  \definecolor{fillcol}{rgb}{0.9,0.9,0.9};
  \pgfsetfillcolor{fillcol}
  \filldraw [opacity=1.0] (117bp,268bp) ellipse (11bp and 12bp);
  \draw (117bp,268bp) node {$\sb\sb$};
\end{scope}
  % Node: ss
\begin{scope}
  \definecolor{strokecol}{rgb}{0.0,0.0,0.0};
  \pgfsetstrokecolor{strokecol}
  \definecolor{fillcol}{rgb}{0.9,0.9,0.9};
  \pgfsetfillcolor{fillcol}
  \filldraw [opacity=1.0] (117bp,319bp) ellipse (11bp and 11bp);
  \draw (117bp,319bp) node {$\sq\sq$};
\end{scope}
  % Node: tt
\begin{scope}
  \definecolor{strokecol}{rgb}{0.0,0.0,0.0};
  \pgfsetstrokecolor{strokecol}
  \definecolor{fillcol}{rgb}{0.9,0.9,0.9};
  \pgfsetfillcolor{fillcol}
  \filldraw [opacity=1.0] (117bp,198bp) ellipse (11bp and 11bp);
  \draw (117bp,198bp) node {$\st\st$};
\end{scope}
  % Node: T5
\begin{scope}
  \definecolor{strokecol}{rgb}{0.0,0.0,0.0};
  \pgfsetstrokecolor{strokecol}
  \draw (351bp,393bp) -- (277bp,393bp) -- (277bp,370bp) -- (351bp,370bp) -- cycle;
  \draw (314bp,381bp) node {\model{T5*} [Fig.~\ref{fig:T5zz}]};
\end{scope}
  % Node: T2
\begin{scope}
  \definecolor{strokecol}{rgb}{0.0,0.0,0.0};
  \pgfsetstrokecolor{strokecol}
  \draw (349bp,352bp) -- (279bp,352bp) -- (279bp,329bp) -- (349bp,329bp) -- cycle;
  \draw (314bp,340bp) node {\model{T2} [Fig.~\ref{fig:T2}]};
\end{scope}
  % Node: T3
\begin{scope}
  \definecolor{strokecol}{rgb}{0.0,0.0,0.0};
  \pgfsetstrokecolor{strokecol}
  \draw (364bp,438bp) -- (264bp,438bp) -- (264bp,415bp) -- (364bp,415bp) -- cycle;
  \draw (314bp,426bp) node {\, \model{T3*} [Figs.~\ref{fig:T3w},\ref{fig:T3C}]\;\;};
\end{scope}
  % Node: T6bbWW
\begin{scope}
  \definecolor{strokecol}{rgb}{0.0,0.0,0.0};
  \pgfsetstrokecolor{strokecol}
  \draw (363bp,149bp) -- (265bp,149bp) -- (265bp,126bp) -- (363bp,126bp) -- cycle;
  \draw (314bp,137bp) node {\model{T6bbWW} [Fig.~\ref{fig:T6bbWW}]};
\end{scope}
  % Node: T1
\begin{scope}
  \definecolor{strokecol}{rgb}{0.0,0.0,0.0};
  \pgfsetstrokecolor{strokecol}
  \draw (349bp,483bp) -- (279bp,483bp) -- (279bp,460bp) -- (349bp,460bp) -- cycle;
  \draw (314bp,471bp) node {\model{T1} [Fig.~\ref{fig:T1}]};
\end{scope}
  % Node: gg
\begin{scope}
  \definecolor{strokecol}{rgb}{0.0,0.0,0.0};
  \pgfsetstrokecolor{strokecol}
  \definecolor{fillcol}{rgb}{0.9,0.9,0.9};
  \pgfsetfillcolor{fillcol}
  \filldraw [opacity=1.0] (117bp,403bp) ellipse (11bp and 11bp);
  \draw (117bp,403bp) node {$\gl\gl$};
\end{scope}
  % Node: T2FVttcc
\begin{scope}
  \definecolor{strokecol}{rgb}{0.0,0.0,0.0};
  \pgfsetstrokecolor{strokecol}
  \draw (363bp,190bp) -- (265bp,190bp) -- (265bp,167bp) -- (363bp,167bp) -- cycle;
  \draw (314bp,178bp) node {\model{T2FVttcc} [Fig.~\ref{fig:T2FVttcc}]};
\end{scope}
  % Node: lsp
\begin{scope}
  \definecolor{strokecol}{rgb}{0.0,0.0,0.0};
  \pgfsetstrokecolor{strokecol}
  \definecolor{fillcol}{rgb}{0.8,0.8,0.8};
  \pgfsetfillcolor{fillcol}
  \filldraw [opacity=1.0] (26bp,268bp) ellipse (25bp and 26bp);
  \draw (26bp,268bp) node {$\chiz$ is LSP};
\end{scope}
  % Node: TChiN2C1
\begin{scope}
  \definecolor{strokecol}{rgb}{0.0,0.0,0.0};
  \pgfsetstrokecolor{strokecol}
  \draw (365bp,67bp) -- (263bp,67bp) -- (263bp,44bp) -- (365bp,44bp) -- cycle;
  \draw (314bp,55bp) node {\model{TChiN2C1} [Fig.~\ref{fig:TChiN2C1}]};
\end{scope}
  % Node: T2tt
\begin{scope}
  \definecolor{strokecol}{rgb}{0.0,0.0,0.0};
  \pgfsetstrokecolor{strokecol}
  \draw (352bp,230bp) -- (276bp,230bp) -- (276bp,207bp) -- (352bp,207bp) -- cycle;
  \draw (314bp,218bp) node {\model{T2tt} [Fig.~\ref{fig:T2tt}]};
\end{scope}
  % Node: T2bb
\begin{scope}
  \definecolor{strokecol}{rgb}{0.0,0.0,0.0};
  \pgfsetstrokecolor{strokecol}
  \draw (354bp,311bp) -- (274bp,311bp) -- (274bp,288bp) -- (354bp,288bp) -- cycle;
  \draw (314bp,299bp) node {\model{T2bb} [Fig.~\ref{fig:T2bb}]};
\end{scope}
\end{small}%
\end{tikzpicture}

\label{fig:smsdecomposition}
\caption{Schematics of the whole SMS decomposition procedure}
\end{figure}


Note, that SMS decomposition can also be performed on the SLHA files directly.
In this note, we chose an SMS decomposition procedure based on LHE files; for a
first document attempt at SMS decomposition this seemed to be the safer
approach. For later studies we envisage employing a SLHA-based SMS decomposition,
in addition to the LHE-based approach.

\FloatBarrier


%%%%%%%%%%%%%%%%%%%%%%%%%%%%%%%%%%%%%%%%
\section{Parameter scan}
%%%%%%%%%%%%%%%%%%%%%%%%%%%%%%%%%%%%%%%%

For the sake of definiteness we scan over a 7 parameter semi-constrained version of the MSSM 
with parameters defined at the GUT scale. 
Concretely, we take a universal gaugino mass parameter $M_{1/2}$,  as well as common scalar mass parameters 
for first and second generation sfermions ($M_{0,1}$) and third generation sfermions ($M_{0,3}$) at $\mgut$.  
We also assume a common trilinear coupling $A_0\equiv A_t=A_b=A_\tau$ at $\mgut$. 
Regarding the Higgs sector,  we trade $M_{H_u}^2$ and $M_{H_d}^2$ at $\mgut$ against 
$\mu$ and $m_A$ at the SUSY scale. Finally, we need the ratio of the two Higgs VEVs, $\tan\beta=v_2/v_1$, 
which we define at the weak scale. 
The scan ranges are given in Table~\ref{tab:scan}.

\begin{table}[t]\centering
\begin{tabular}{|c|c|c|c|c|c|c|c|}
  \hline
  $M_{1/2}$ & $M_{0,1}$ & $M_{0,3}$ & $A_0$ & $m_A$ & $\mu$ & $\tan\beta$ \\ 
  \hline
  0.1\,--\,1 & 0\,--\,3 & 0\,--\,1 & $-3$\,--\,3 & 0.1\,--\,2 & 0.1\,--\,1 & 2\,--\,50 \\
  \hline
\end{tabular}
\caption{Scan ranges used in this study. Dimensionful quantities are in TeV units.}
\label{tab:scan}
\end{table}

\marginpar{\bf add versions and refs}
We use SOFTSUSY for the computation of the masses and mixings,  SUPERISO for flavor observables, 
micrOMEGAs for dark matter observables and particle decay tables, and HDECAY for computing Higgs decays.
For the SM input values, we use $m_t=....$, $m_b(m_b)=4.19$~GeV, and $\alpha_s(M_Z)=0.1187$ ????  
We require a neutralino LSP.  
The constraints imposed are listed in Table~\ref{tab:constraints}.

\begin{table}[t]\centering
\begin{tabular}{|c|c|c|c|c|}
\hline
  \bf Observable  & \bf Experimental result   & \bf TH uncertainty & \bf Constraint imposed \\
\hline
 BR$(b \rightarrow s\gamma) $ & $(3.55 \pm 0.23 \pm 0.09)\times 10^{-4}$ \cite{Amhis:2012bh} & $0.24 \times 10^{-4}$ & $[2.86,\,4.24] \times 10^{-4}$  \\
\hline
BR$(B_s \rightarrow \mu^+ \mu^-)$ & $3.2^{+1.5}_{-1.2} \times 10^{-9}$ \cite{lhcb:2012ct} & 10\% & $[0.7,\,6.3] \times 10^{-9}$\\
\hline
$\Delta a_\mu$ & $(26.1 \pm 8.0)\times 10^{-10}$ \cite{Hagiwara:2011af} & $10\times 10^{-10}$ & $[-0.5,\,51.7] \times 10^{-10}$ \\
\hline
$\Omega h^2$ & $0.1123\pm 0.0055$~\cite{} & 10\% & $<0.136$ \\
\hline
$m_h$ & $125.2\pm 0.3\pm 0.6$ GeV (ATLAS) & 3~GeV \cite{xxx} & $125.5\pm 3$~GeV \\
              & $125.8\pm 0.4\pm 0.4$ GeV (CMS)\phantom{00} & &  \\
\hline
 sparticle masses & LEP  & --- & (micrOMEGAs) \\
\hline
\end{tabular}
\caption{Constraints used to define the valid parameter space before applying the SMS limits.}
\label{tab:constraints}
\end{table}



%%%%%%%%%%%%%%%%%%%%%%%%%%%%%%%%%%%%%%%%
\section{Results}
%%%%%%%%%%%%%%%%%%%%%%%%%%%%%%%%%%%%%%%%


%%%%%%%%%%%%%%%%%%%%%%%%%%%%%%%%%%%%%%%%
\section{Conclusions}
%%%%%%%%%%%%%%%%%%%%%%%%%%%%%%%%%%%%%%%%


\clearpage
%%%%%%%%%%%%%%%%%%%%%%%%%%%%%%%%%%%%%%%%
\section*{Acknowledgements} 
%%%%%%%%%%%%%%%%%%%%%%%%%%%%%%%%%%%%%%%%

This work was supported in part by the by the French 
ANR project {\sc DMAstroLHC}. %BLAN12-???.

\begin{appendix}
\FloatBarrier

\section{Appendix: The SMS cases}
\label{app:smscases}
In this appendix, details about the SMS cases are given, split up by the
production mechanism.

\subsection{The ``gluinos'' case}
\label{ssec:gluinocase}
The case of gluino-gluino production is the richest in terms of experimental results:
Fig.~\ref{fig:gluinoSMSes} shows the Feynman graphs of all processes that can be covered by LHC results, Tab.~\ref{tab:classgluinos} lists how all SMSes are identified, and what results have been applied.

\begin{figure}[h!t]
\begin{center}
\begin{tabular}{lcr}
\subfigure[\label{fig:T1}T1]{\includegraphics[width=0.2\linewidth]{figures/T1_feyn.pdf}}\spacer &
\subfigure[\label{fig:T3w}T3w]{\includegraphics[width=0.26\linewidth]{figures/T3w_feyn.pdf}}\spacer &
\subfigure[\label{fig:T3C}T3C]{\includegraphics[width=0.26\linewidth]{figures/T3C_feyn.pdf}}\spacer \\
\subfigure[\label{fig:T5zz}T5zz]{\includegraphics[width=0.26\linewidth]{figures/T5zz_feyn.pdf}} \spacer &
\subfigure[\label{fig:T5qqqq}T5qqqq]{\includegraphics[width=0.26\linewidth]{figures/T5zz_feyn.pdf}} \spacer \\
% \subfigure[\label{fig:T5ww}T5ww]{\includegraphics[width=0.2\linewidth]{figures/T3N_feyn.pdf}} \spacer \\
\end{tabular}
\caption{A few simplified models for gluino-gluino production. \fixme{Once we know what models we use, we choose more appropriate models. need to (re)draw a few plots: T3C, T5qqqq}}
\label{fig:gluinoSMSes}
\end{center}
\end{figure}

\begin{table}[h!]\centering
\begin{tabular}{|c|c|c|c|c|c|}
\hline
additional & & & & & Results \\
particle content & Production & Decay & SMS & Figure & from \\
\hline
qqqq & $\gl \rightarrow qq\chiz$ & & \model{T1} & Fig.~\ref{fig:T1} & \AlphaT, \HsTjets, \MTtwo, razor \\
\hline
qqqqW & $\gl \rightarrow qq\chiz$, & $\chipm\rightarrow W \chiz$  & \model{T3w}  & Fig.~\ref{fig:T3w}  & $\emu$ LS, $\emu$ LP, $\emu$ ANN  \\
 & $\gl \rightarrow qq\chipm$ & & & & \\
\hline
qqqql & $\gl \rightarrow qq\chiz$, & $\chipm\rightarrow l \nu \chiz$  & \model{T3C}  & Fig.~\ref{fig:T3C}  & $ \emu$ LS, $\emu$ LP, $\emu$ ANN \\
 & $\gl \rightarrow qq\chipm$ & & & & hadronic \model{T3w} results apply \\
\hline
qqqqZZ& $\gl \rightarrow qq\chitz$ & $\chitz\rightarrow Z \chiz$  & \model{T5zz} & Fig.~\ref{fig:T5zz} & \AlphaT, \HsTjets, \MTtwo \\
 & & &                     & & \ZMET, JZB, multileptons  \\
\hline
qqqqqqqq& $\gl \rightarrow qqqq\chiz$ & & \model{T5qqqq} & Fig.~\ref{fig:T5qqqq} & \AlphaT, \HsTjets, \MTtwo  \\
 & & &                     & & hadronic \model{T5zz} results apply\\
\hline
\end{tabular}
\caption{SMS case ``gluinos'': Classification of events with two gluinos as
SUSY mothers. ``q'' as additional particle content means any \first or \second generation quark.}
\label{tab:classgluinos}
\end{table}

\FloatBarrier

\subsection{The ``squarks'' case}
\label{ssec:squarkcase}

\begin{figure}[ht!]
\begin{center}
%\begin{tabular}{lcr}
\begin{tabular}{c}
\subfigure[\label{fig:T2}T2]{\includegraphics[width=0.2\linewidth]{figures/T2_feyn.pdf}}\spacer
%\subfigure[\label{fig:T2tt}T2tt]{\includegraphics[width=0.2\linewidth]{figures/T2tt_feyn.pdf}}\spacer &
%\subfigure[\label{fig:T2bb}T2bb]{\includegraphics[width=0.2\linewidth]{figures/T2bb_feyn.pdf}}\spacer \\
\end{tabular}
\caption{A few simplified models for squark-squarkbar production. \fixme{Once we know what models we use, we choose more appropriate models.}}
\label{fig:squarkSMSes}
\end{center}
\end{figure}


\begin{table}[h!]\centering
\begin{tabular}{|c|c|c|c|c|c|}
\hline
additional & & & & & Results \\
particle content & Production & Decay & SMS & Figure & from \\
\hline
qq & $\sq \rightarrow q\chiz$ & & \model{T2} & Fig.~\ref{fig:T2} & \AlphaT, \HsTjets, razor \\
\hline
\end{tabular}
\caption{SMS case ``squarks'': Classification of events with two first or second generatiorn squarks as
SUSY mothers. ``q'' as additional particle content means any \first or \second generation quark.}
\label{tab:classsquarks}
\end{table}

\FloatBarrier

\subsection{The \third generation case}
\label{ssec:thirdcase}
Naturalness demands that both stop quarks and one sbottom quark be very light, typically below 1 TeV (\fixme{what can we cite here?}).
Both ATLAS and CMS have therefore invested major fractions of their ressources into
probing the third generation of the squark spectrum, particularly in the last two years.
Thus, a fair number of SMS results for stop and sbottom production models is
available, as can be seen in Fig.~\ref{fig:thirdSMSes}, Tab.~\ref{tab:classstops},
and Tab.~\ref{tab:classsbottoms}.

\begin{figure}[ht!]
\begin{center}
\begin{tabular}{lcr}
\subfigure[\label{fig:T2tt}T2tt]{\includegraphics[width=0.2\linewidth]{figures/T2tt_feyn.pdf}}\spacer &
\subfigure[\label{fig:T6bbWW}T6bbWW]{\includegraphics[width=0.26\linewidth]{figures/T6bbWW_feyn.pdf}}\spacer &
\subfigure[\label{fig:T2FVttcc}T2FVttcc]{\includegraphics[width=0.2\linewidth]{figures/T2FVttcc_feyn.pdf}}\spacer \\
\subfigure[\label{fig:T2bb}T2bb]{\includegraphics[width=0.2\linewidth]{figures/T2bb_feyn.pdf}}\spacer &
\subfigure[\label{fig:T6ttWW}T6ttWW]{\includegraphics[width=0.26\linewidth]{figures/T6ttWW_feyn.pdf}}\spacer \\
\end{tabular}
\caption{A few simplified models for stop and sbottom production. \fixme{Once we know what models we use, we redo this list.}}
\label{fig:thirdSMSes}
\end{center}
\end{figure}


\begin{table}[h!]\centering
\begin{tabular}{|c|c|c|c|c|c|}
\hline
additional & & & & & Results \\
particle content & Production & Decay & SMS & Figure & from \\
\hline
tt & $\sTop \rightarrow t\chiz$ & & \model{T2tt} & Fig.~\ref{fig:T2tt} & razor, razor+b, razor+jets,  \\
                           & & & & & \AlphaT, hadronic \sTop, leptonic \sTop, \\
                           & & & & & \ATLLepStop \\
\hline
$bb\chipm\chipm$WW & $\sTop \rightarrow b\chipm$ & $\chipm \rightarrow W \chiz$ & \model{T6bbWW} & Fig.~\ref{fig:T6bbWW} & \ATLLepStop, \ATLHadStop \\
\hline
cc & $\sTop \rightarrow c\chiz$ & & \model{T2FVttcc} & Fig.~\ref{fig:T2FVttcc} & alphaT8TeV \\
   &  & &    & & monojets8, razormono8 \\
\hline
\end{tabular}
\caption{SMS case ``stops'': Classification of events with two top squarks as SUSY mothers}
\label{tab:classstops}
\end{table}

\begin{table}[h!]\centering
\begin{tabular}{|c|c|c|c|c|c|}
\hline
additional & & & & & Results \\
particle content & Production & Decay & SMS & Figure & from \\
\hline
bb & $\sBottom \rightarrow b\chiz$ & & \model{T2bb} & Fig.~\ref{fig:T2bb} & razor+b, \AlphaT,alphaT8 \\
\hline
ttWW & $\sBottom \rightarrow t\chipm$ & $\chipm \rightarrow W \chiz$ & \model{T6ttWW} & Fig.~\ref{fig:T6ttWW} & SS+b \\
\hline
\end{tabular}
\caption{SMS case ``sbottoms'': Classification of events with two bottom squarks as SUSY mothers}
\label{tab:classsbottoms}
\end{table}

\FloatBarrier

\subsection{Weakino production}
\label{ssec:weakinocase}
In the pMSSM, production of charginos and/or heavy neutralinos results in
cascades with only little energy; they can thus only be probed by multi-lepton
signatures, no hadronic results are available for such small mass splittings.
See Fig.~\ref{fig:weakinoSMSes} for the Feynman graphs and
Tab.~\ref{tab:classweakinos} for a list of the available results.

\begin{figure}[ht!]
\begin{center}
\begin{tabular}{lcr}
\subfigure[\label{fig:TChiwz}TChiwz]{\includegraphics[width=0.2\linewidth]{figures/TChiwz_feyn.pdf}}\spacer &
\subfigure[\label{fig:TChizz}TChizz]{\includegraphics[width=0.2\linewidth]{figures/TChizz_feyn.pdf}}\spacer &
\subfigure[\label{fig:TChiN2C1}TChiN2C1]{\includegraphics[width=0.2\linewidth]{figures/TChiN2C1_feyn.pdf}}\spacer \\
\subfigure[\label{fig:TChiChipmSlepSlep}TChiChipmSlepSlep]{\includegraphics[width=0.26\linewidth]{figures/TChiChipmSlepSlep_feyn.pdf}}\spacer &
\subfigure[\label{fig:TChiChipmSnuSlep}TChiChipmSnuSlep]{\includegraphics[width=0.26\linewidth]{figures/TChiChipmSnuSlep_feyn.pdf}}\spacer \\
\end{tabular}
\caption{The simplified models used for weakino production. \fixme{Once we know what models we use, we redo this list.}}
\label{fig:weakinoSMSes}
\end{center}
\end{figure}



\begin{table}[h!]\centering
\begin{tabular}{|c|c|c|c|c|c|}
\hline
additional & & & & & Results \\
particle content & Production & Decay & SMS & Figure & from \\
\hline
WZ & $\chipm\rightarrow W\chiz$ & & \model{TChiwz} & Fig.~\ref{fig:TChiwz} & multi leptons, \\
 & $\chitz \rightarrow Z\chiz$ & & & & comb. leptons \\
\hline
lll & $\chipm\rightarrow l \nu\chiz$ & & \model{TChiN2C1} & Fig.~\ref{fig:TChiN2C1} & multi leptons, \\
 & $\chitz \rightarrow ll\chiz$ & & & & comb. leptons \\
\hline
ZZ & $\chitz \rightarrow Z\chiz$ & & \model{TChizz} & Fig.~\ref{fig:TChizz} & ... \\
\hline
$\slep\slep$lll & $\chipm\rightarrow \nu\slep, $ & $\slep\rightarrow l\chiz$ & \model{TChiChipm-} & Fig.~\ref{fig:TChiChipmSlepSlep} & multi leptons, \\
 & $\chitz \rightarrow l\slep$ & & \model{SlepSlep} & & comb. leptons \\
\hline
$\slep\snu$lll  & $\chipm\rightarrow l\snu, \chitz \rightarrow l\slep$ & $\slep\rightarrow l\chiz$  & \model{TChiChipm-} & Fig.~\ref{fig:TChiChipmSnuSlep} & multi leptons, \\
 & & $\snu\rightarrow \nu\chiz$  & \model{SnuSlep} & & comb. leptons \\
\hline
\end{tabular}
\caption{SMS case ``weakinos'': Classification of events with charginos or neutralinos as SUSY mothers. Add TChiN2C1 et al.}
\label{tab:classweakinos}
\end{table}

\FloatBarrier

\subsection{Decomposing a pMSSM point -- an example}
\label{ssec:decomposeexample}
\fixme{I am only starting this section.} \\

Point 5jc0G1ixS2wq98jT76e8ieLS73LcJSWAE5FGb2ZFwUqx7xY6
\begin{figure}[ht!]
\begin{center}
\begin{tabular}{lc}
\subfigure[\label{fig:ruler}ruler plot, masses are given in GeV.]
{\includegraphics[width=0.26\linewidth]{figures/ruler.pdf}}\spacer &
\subfigure[\label{fig:decayplot}decay plot]{\includegraphics[width=0.72\linewidth]{figures/decay.pdf}}
\end{tabular}
\caption{ \fixme{Starting to add plots that show the decomposition.}}
\label{fig:decompositionexample}
\end{center}
\end{figure}

Blahblah walkding: \\
\url{http://www.hephy.at/user/walten/cgi-bin/pmssm.py?\\key=5jc0G1ixS2wq98jT76e8ieLS73LcJSWAE5FGb2ZFwUqx7xY6}
\FloatBarrier


\end{appendix}


\bibliography{references}

\end{document}

