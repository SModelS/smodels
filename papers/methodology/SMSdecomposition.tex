\subsection{SMS decomposition: The procedure}
\label{ssec:decomp}

Starting point of the SMS decomposition is an SLHA file which contains the
mass spectrum and decay table of a specific MSSM parameter point.
Next, the SLHA file is passed to pythia to generate 10000 events. 
As an output format we chose the LHE file format.
Per event, we then count the SUSY particles, with the SUSY ``mothers'' (i.e.
the first SUSY particles in the decay chains) playing a special role.
Assuming $\chiz$ to always be the LSP  and taking into account conservation of
R-parity, we found the categorization scheme given in
Tab.~\ref{tab:classmothers} to be sufficient. 
Within these categories the events are further categorized based on the 
remaining SUSY particle content. Tabs.~\ref{tab:classgluinos},\ref{tab:classsquarks},\ref{tab:classstops},\ref{tab:classsbottoms},\ref{tab:classweakinos} 
discuss these subcategories. The events' requirements on the particle content is listed, in order for the event to be classified as a specific simplified model.
The requirement on the SUSY mothers and the requirement on the LSPs are not listed.
Fig.~\ref{fig:smsdecomposition} shows an overview of the whole procedure of SMS decomposition.

\begin{table}[h!t]\centering
\begin{tabular}{|c|c|c|}
\hline
Requirement on  & SMS & discussed \\
SUSY mothers  & case  & in Sec. \\
\hline
two gluinos & ``gluinos'' & Sec.~\ref{ssec:gluinocase} \\
two squarks (\first, \second gen) & ``squarks'' & Sec.~\ref{ssec:squarkcase} \\
one gluino, one squark (any generation) & ``associate'' & not discussed \\
two stops & ``stops'' & Sec.~\ref{ssec:thirdcase} \\
two sbottoms & ``sbottoms'' & Sec.~\ref{ssec:thirdcase} \\
two weakinos & ``weakinos'' & Sec.~\ref{ssec:weakinocase} \\
two sleptons & ``sleptons'' & not discussed \\
\hline
\end{tabular}
\caption{Classification of events according to SUSY mothers}
\label{tab:classmothers}
\end{table}

\begin{figure}[h!t]\centering
% \begin{tikzpicture}[anchor=mid,>=latex',line join=bevel,]
\begin{tikzpicture}[>=latex',line join=bevel,]
  \pgfsetlinewidth{1bp}
\begin{small}%
\pgfsetcolor{black}
  % Edge: gg -> T5
  \draw [->] (126.57bp,397.05bp) .. controls (135.39bp,391.92bp) and (149.05bp,384.92bp)  .. (162bp,382bp) .. controls (196.2bp,374.3bp) and (235.83bp,374.32bp)  .. (276.33bp,376.82bp);
  \definecolor{strokecol}{rgb}{0.0,0.0,0.0};
  \pgfsetstrokecolor{strokecol}
  \draw (195bp,393bp) node {qqqqV$^{(*)}$V$^{(*)}$};
  % Edge: tt -> T2tt
  \draw [->] (128.16bp,199.13bp) .. controls (154.04bp,201.76bp) and (219.44bp,208.4bp)  .. (275.35bp,214.08bp);
  \draw (195bp,218bp) node {tt};
  % Edge: ss -> T2
  \draw [->] (128.16bp,320.19bp) .. controls (154.72bp,323.02bp) and (222.91bp,330.29bp)  .. (278.9bp,336.26bp);
  \draw (195bp,339bp) node {qq};
  % Edge: lsp -> nn
  \draw [->] (36.796bp,245.22bp) .. controls (54.879bp,207.07bp) and (90.966bp,130.93bp)  .. (112.28bp,85.962bp);
  % Edge: bb -> T6ttWW
  \draw [->] (128.38bp,265.1bp) .. controls (137.42bp,262.95bp) and (150.43bp,260.18bp)  .. (162bp,259bp) .. controls (192.86bp,255.84bp) and (227.43bp,255.37bp)  .. (265.95bp,255.98bp);
  \draw (195bp,268bp) node {ttWW};
  % Edge: lsp -> gg
  \draw [->] (40.219bp,289.09bp) .. controls (58.037bp,315.53bp) and (88.262bp,360.37bp)  .. (110.67bp,393.61bp);
  % Edge: bb -> T2bb
  \draw [->] (128.44bp,270.46bp) .. controls (137.51bp,272.38bp) and (150.54bp,275.03bp)  .. (162bp,277bp) .. controls (195.61bp,282.79bp) and (233.59bp,288.29bp)  .. (273.26bp,293.71bp);
  \draw (195bp,295bp) node {bb};
  % Edge: nn -> TChiwz
  \draw [->] (143.16bp,78.656bp) .. controls (172.34bp,81.618bp) and (220.39bp,86.497bp)  .. (268.46bp,91.377bp);
  \draw (195bp,96bp) node {WZ};
  % Edge: tt -> T6bbWW
  \draw [->] (123.36bp,188.54bp) .. controls (131.1bp,177.87bp) and (145.25bp,160.99bp)  .. (162bp,153bp) .. controls (190.39bp,139.46bp) and (225.07bp,135.03bp)  .. (264.16bp,134.01bp);
  \draw (195bp,162bp) node {bbWW};
  % Edge: lsp -> ss
  \draw [->] (48.494bp,280.61bp) .. controls (63.718bp,289.14bp) and (83.605bp,300.28bp)  .. (107.01bp,313.4bp);
  % Edge: gg -> T1
  \draw [->] (124.4bp,411.38bp) .. controls (132.67bp,420.18bp) and (146.89bp,433.74bp)  .. (162bp,441bp) .. controls (195.65bp,457.17bp) and (237.5bp,464.63bp)  .. (278.93bp,469.08bp);
  \draw (195bp,470bp) node {qqqq};
  % Edge: lsp -> bb
  \draw [->] (51.834bp,268bp) .. controls (65.489bp,268bp) and (82.071bp,268bp)  .. (105.45bp,268bp);
  % Edge: nn -> TChiN2C1
  \draw [->] (133.23bp,67.857bp) .. controls (141.6bp,64.088bp) and (152.1bp,60.007bp)  .. (162bp,58bp) .. controls (191.37bp,52.042bp) and (224.6bp,50.779bp)  .. (262.48bp,51.452bp);
  \draw (195bp,67bp) node {lll};
  % Edge: lsp -> tt
  \draw [->] (46.196bp,252.46bp) .. controls (62.304bp,240.07bp) and (84.615bp,222.91bp)  .. (107.92bp,204.98bp);
  % Edge: gg -> T3
  \draw [->] (128.16bp,404.3bp) .. controls (151.79bp,407.06bp) and (208.4bp,413.67bp)  .. (263.15bp,420.06bp);
  \draw (195bp,426bp) node {qqqqV$^{(*)}$};
  % Edge: nn -> TChiSlep
  \draw [->] (122.82bp,65.896bp) .. controls (130.25bp,54.068bp) and (144.28bp,34.959bp)  .. (162bp,26bp) .. controls (184.46bp,14.642bp) and (211.16bp,9.6441bp)  .. (245.96bp,7.2855bp);
  \draw (195bp,36.5bp) node {\slep$|$\snu lll};
  % Edge: tt -> T2FVttcc
  \draw [->] (127.11bp,193.37bp) .. controls (136.07bp,189.51bp) and (149.61bp,184.31bp)  .. (162bp,182bp) .. controls (192.05bp,176.41bp) and (226.01bp,175.06bp)  .. (264.3bp,175.39bp);
  \draw (195bp,190bp) node {cc};
  % Node: TChiSlep
\begin{scope}
  \definecolor{strokecol}{rgb}{0.0,0.0,0.0};
  \pgfsetstrokecolor{strokecol}
  \draw (382bp,22bp) -- (246bp,23bp) -- (246bp,0bp) -- (382bp,0bp) -- cycle;
  \draw (314bp,11bp) node {\;\;\;\;\model{TChiSlep*} [Figs.~\ref{fig:TChiChipmSlepSlep},\ref{fig:TChiChipmSnuSlep}] \;\;};
\end{scope}
  % Node: T6ttWW
\begin{scope}
  \definecolor{strokecol}{rgb}{0.0,0.0,0.0};
  \pgfsetstrokecolor{strokecol}
  \draw (362bp,270bp) -- (266bp,270bp) -- (266bp,247bp) -- (362bp,247bp) -- cycle;
  \draw (314bp,258bp) node {\model{T6ttWW} [Fig.~\ref{fig:T6ttWW}]};
\end{scope}
  % Node: TChiwz
\begin{scope}
  \definecolor{strokecol}{rgb}{0.0,0.0,0.0};
  \pgfsetstrokecolor{strokecol}
  \draw (359bp,108bp) -- (269bp,108bp) -- (269bp,85bp) -- (359bp,85bp) -- cycle;
  \draw (314bp,96bp) node {\model{TChiwz} [Fig.~\ref{fig:TChiwz}]};
\end{scope}
  % Node: nn
\begin{scope}
  \definecolor{strokecol}{rgb}{0.0,0.0,0.0};
  \pgfsetstrokecolor{strokecol}
  \definecolor{fillcol}{rgb}{0.9,0.9,0.9};
  \pgfsetfillcolor{fillcol}
  \filldraw [opacity=1.0] (117bp,76bp) ellipse (27bp and 10bp);
  \draw (117bp,76bp) node {weakino};
\end{scope}
  % Node: bb
\begin{scope}
  \definecolor{strokecol}{rgb}{0.0,0.0,0.0};
  \pgfsetstrokecolor{strokecol}
  \definecolor{fillcol}{rgb}{0.9,0.9,0.9};
  \pgfsetfillcolor{fillcol}
  \filldraw [opacity=1.0] (117bp,268bp) ellipse (11bp and 12bp);
  \draw (117bp,268bp) node {$\sb\sb$};
\end{scope}
  % Node: ss
\begin{scope}
  \definecolor{strokecol}{rgb}{0.0,0.0,0.0};
  \pgfsetstrokecolor{strokecol}
  \definecolor{fillcol}{rgb}{0.9,0.9,0.9};
  \pgfsetfillcolor{fillcol}
  \filldraw [opacity=1.0] (117bp,319bp) ellipse (11bp and 11bp);
  \draw (117bp,319bp) node {$\sq\sq$};
\end{scope}
  % Node: tt
\begin{scope}
  \definecolor{strokecol}{rgb}{0.0,0.0,0.0};
  \pgfsetstrokecolor{strokecol}
  \definecolor{fillcol}{rgb}{0.9,0.9,0.9};
  \pgfsetfillcolor{fillcol}
  \filldraw [opacity=1.0] (117bp,198bp) ellipse (11bp and 11bp);
  \draw (117bp,198bp) node {$\st\st$};
\end{scope}
  % Node: T5
\begin{scope}
  \definecolor{strokecol}{rgb}{0.0,0.0,0.0};
  \pgfsetstrokecolor{strokecol}
  \draw (351bp,393bp) -- (277bp,393bp) -- (277bp,370bp) -- (351bp,370bp) -- cycle;
  \draw (314bp,381bp) node {\model{T5*} [Fig.~\ref{fig:T5zz}]};
\end{scope}
  % Node: T2
\begin{scope}
  \definecolor{strokecol}{rgb}{0.0,0.0,0.0};
  \pgfsetstrokecolor{strokecol}
  \draw (349bp,352bp) -- (279bp,352bp) -- (279bp,329bp) -- (349bp,329bp) -- cycle;
  \draw (314bp,340bp) node {\model{T2} [Fig.~\ref{fig:T2}]};
\end{scope}
  % Node: T3
\begin{scope}
  \definecolor{strokecol}{rgb}{0.0,0.0,0.0};
  \pgfsetstrokecolor{strokecol}
  \draw (364bp,438bp) -- (264bp,438bp) -- (264bp,415bp) -- (364bp,415bp) -- cycle;
  \draw (314bp,426bp) node {\, \model{T3*} [Figs.~\ref{fig:T3w},\ref{fig:T3C}]\;\;};
\end{scope}
  % Node: T6bbWW
\begin{scope}
  \definecolor{strokecol}{rgb}{0.0,0.0,0.0};
  \pgfsetstrokecolor{strokecol}
  \draw (363bp,149bp) -- (265bp,149bp) -- (265bp,126bp) -- (363bp,126bp) -- cycle;
  \draw (314bp,137bp) node {\model{T6bbWW} [Fig.~\ref{fig:T6bbWW}]};
\end{scope}
  % Node: T1
\begin{scope}
  \definecolor{strokecol}{rgb}{0.0,0.0,0.0};
  \pgfsetstrokecolor{strokecol}
  \draw (349bp,483bp) -- (279bp,483bp) -- (279bp,460bp) -- (349bp,460bp) -- cycle;
  \draw (314bp,471bp) node {\model{T1} [Fig.~\ref{fig:T1}]};
\end{scope}
  % Node: gg
\begin{scope}
  \definecolor{strokecol}{rgb}{0.0,0.0,0.0};
  \pgfsetstrokecolor{strokecol}
  \definecolor{fillcol}{rgb}{0.9,0.9,0.9};
  \pgfsetfillcolor{fillcol}
  \filldraw [opacity=1.0] (117bp,403bp) ellipse (11bp and 11bp);
  \draw (117bp,403bp) node {$\gl\gl$};
\end{scope}
  % Node: T2FVttcc
\begin{scope}
  \definecolor{strokecol}{rgb}{0.0,0.0,0.0};
  \pgfsetstrokecolor{strokecol}
  \draw (363bp,190bp) -- (265bp,190bp) -- (265bp,167bp) -- (363bp,167bp) -- cycle;
  \draw (314bp,178bp) node {\model{T2FVttcc} [Fig.~\ref{fig:T2FVttcc}]};
\end{scope}
  % Node: lsp
\begin{scope}
  \definecolor{strokecol}{rgb}{0.0,0.0,0.0};
  \pgfsetstrokecolor{strokecol}
  \definecolor{fillcol}{rgb}{0.8,0.8,0.8};
  \pgfsetfillcolor{fillcol}
  \filldraw [opacity=1.0] (26bp,268bp) ellipse (25bp and 26bp);
  \draw (26bp,268bp) node {$\chiz$ is LSP};
\end{scope}
  % Node: TChiN2C1
\begin{scope}
  \definecolor{strokecol}{rgb}{0.0,0.0,0.0};
  \pgfsetstrokecolor{strokecol}
  \draw (365bp,67bp) -- (263bp,67bp) -- (263bp,44bp) -- (365bp,44bp) -- cycle;
  \draw (314bp,55bp) node {\model{TChiN2C1} [Fig.~\ref{fig:TChiN2C1}]};
\end{scope}
  % Node: T2tt
\begin{scope}
  \definecolor{strokecol}{rgb}{0.0,0.0,0.0};
  \pgfsetstrokecolor{strokecol}
  \draw (352bp,230bp) -- (276bp,230bp) -- (276bp,207bp) -- (352bp,207bp) -- cycle;
  \draw (314bp,218bp) node {\model{T2tt} [Fig.~\ref{fig:T2tt}]};
\end{scope}
  % Node: T2bb
\begin{scope}
  \definecolor{strokecol}{rgb}{0.0,0.0,0.0};
  \pgfsetstrokecolor{strokecol}
  \draw (354bp,311bp) -- (274bp,311bp) -- (274bp,288bp) -- (354bp,288bp) -- cycle;
  \draw (314bp,299bp) node {\model{T2bb} [Fig.~\ref{fig:T2bb}]};
\end{scope}
\end{small}%
\end{tikzpicture}

\label{fig:smsdecomposition}
\caption{Schematics of the whole SMS decomposition procedure}
\end{figure}


Note, that SMS decomposition can also be performed on the SLHA files directly.
In this note, we chose an SMS decomposition procedure based on LHE files; for a
first document attempt at SMS decomposition this seemed to be the safer
approach. For later studies we envisage employing a SLHA-based SMS decomposition,
in addition to the LHE-based approach.

\FloatBarrier
